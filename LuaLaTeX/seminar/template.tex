% Version: 20241221

% プリアンブル (TeX ドキュメントの設定)=====

% ドキュメントクラスの設定: 左寄せ, A4用紙, 2段組
\documentclass[fleqn, a4paper, twocolumn]{ltjsarticle}

% パッケージの読み込み
\usepackage[a-2u]{pdfx}
\usepackage[dvipsnames]{xcolor}   % 様々な色を使えるようにするパッケージ
\usepackage[luatex]{graphicx}   % 図とかを入れられるようにするパッケージ
\usepackage{ascmac}               % 枠やボックスを作成するためのパッケージ
\usepackage{CJKutf8}	      % 日本語を使えるようにするパッケージ
\usepackage{luatex85}	     % LuaTeXのバージョンによるエラーを回避
\usepackage{multicol} % 複数カラムのレイアウトをサポート
\usepackage{physics, amssymb, amsmath} % 数式や物理表現をサポートするパッケージ群
% physics: 物理記号(例: 内積、偏微分)
% amssymb: 数学記号(例: 集合、特殊な矢印)
% amsmath: 高度な数式表現(例: 行列、分数)
\usepackage{enumitem} % 箇条書きやリストのカスタマイズ
\usepackage{titlesec} % 見出しのデザイン変更
\usepackage{titling} % タイトルのカスタマイズ
\usepackage{here} % 図表の強制的な配置制御([H]オプションを使用可能)
\usepackage{subcaption} % 図の中で複数の小図(サブキャプション)を管理
\usepackage{fancyhdr} % ヘッダー・フッターのカスタマイズ
\usepackage[deluxe]{luatexja-preset} % LuaTeX-jaの設定


% TikZの設定 ---
\usepackage{tikz, braket} % TikZ: 図やグラフの作成
% braket: 量子力学のブラケット記法をサポート
\usetikzlibrary{patterns, intersections, calc, quotes, angles, arrows.meta, through, shapes.geometric} 
% patterns: 塗りつぶしパターン
% intersections: 線の交点計算
% calc: 座標計算
% quotes: 図中の角度や距離の注記
% angles: 角度の描画
% arrows.meta: 矢印のスタイル拡張
% through: 特定点を通る形状の描画
% shapes.geometric: 基本的な幾何図形の描画
\usetikzlibrary{shapes, positioning} 
% shapes: さらに多様な図形を描画
% positioning: 図形間の位置関係を制御
% ---

% 余白の設定 ---
\setlength{\mathindent}{3zw} % 数式のインデントを設定
\setlength{\topmargin} {0cm} % 上マージンをゼロに
\iftombow
    \addtolength{\topmargin}{0cm} % トンボ付きの場合の調整
\else
    \addtolength{\topmargin}{-1truein} % トンボなしの場合の調整
\fi
\setlength{\textheight}   {45\baselineskip} % テキストの高さ(行基準)
\addtolength{\textheight} {\topskip} % 行の余白分を追加
\setlength{\oddsidemargin}{-15pt} % 偶数ページの余白
\setlength{\textwidth}    {50zw} % テキストの幅(文字単位)
% ---

% ページスタイルの設定 ---
\pagestyle{empty} % ヘッダー・フッターを非表示
\lhead{} % 左ヘッダー空白
\chead{} % 中央ヘッダー空白
\lfoot{} % 左フッター空白
\cfoot{--\;\thepage\;--} % ページ番号を中央に表示
\rfoot{} % 右フッター空白
\allowdisplaybreaks % 数式がページをまたぐことを許可
% ---

% 図番号などの設定 ---
\numberwithin{figure}{section}
\numberwithin{table}{section}
\numberwithin{equation}{section}

\newcommand{\erfc}{\mathrm{erfc}}
% ---

% 資料のタイトルを入れる
\title{ゼミ用テンプレート}
% 資料の著者名を入れる
\date{2025年04月01日}
% ゼミの日を入れる
\author{名字 名前}

% =====

\begin{document}

% タイトル部分 (操作不要)---
\noindent{\textgt{\Large \thetitle}}\\  % 資料のタイトル
\noindent \theauthor \hfill             % 資料の著者名
\thedate                                % ゼミの日
% ---

\section{前回ゼミまでの進捗状況}
前回のゼミ資料を内容をまとめる.

\subsection{前回ゼミでの指摘事項}
前回のゼミで指摘された事項を書く.
箇条書きの場合は以下のようにする.
\begin{itemize}
	\item 現象について定量的な説明をすべき
	\item 理論値を示せ
\end{itemize}

\subsection{前回資料の補足・訂正など}
前回のゼミ資料の誤記などがあれば訂正する.なければこのsubsectionは削除.

\section{導入と基礎知識,原理など}
今回の報告内容で,基礎理論等説明したほうが良い場合は
ここに書く.

\subsection{箇条書きの例}
見やすさのため,ある程度内容が多ければこのようにsubsectionに分けたほうが良い.
フローなどを説明する際に,番号付きの箇条書きをしたい場合は以下のように記述できる.
\begin{enumerate}
	\item 一次変調
	\item IFFT
	\item Cyclic Prefix付加
\end{enumerate}
または箇条書きのようにして用語を説明する場合,以下のように記述できる.
\begin{description}
	\item [QAM]: Quadrature Amplitude Modulation
	\item [PSK]: Phase Shift Keying
\end{description}

\subsection{数式の例}
数式の例 (1行で済む場合)
\begin{equation}
	\exp \qty[\mathrm{j}2\pi ft] = \cos \qty(2\pi ft) + \mathrm{j}\sin \qty(2\pi ft)
	\label{eq: euler}
\end{equation}

数式の例 (1行ではおさまらない場合)
\begin{equation}
	\begin{split}
		\cos 2\theta & = \cos^2\theta-\sin^2\theta \\ % &は位置を合わせるための記号
		             & = 2\cos^2\theta - 1         \\         % &の後ろの文字で位置が合うので,基本的には
		             & = 1 - 2\sin^2\theta         % イコールで合わせると良い.
		\label{eq: doubleangle}
	\end{split}
\end{equation}

数式の例 (複数の式を列挙したい場合)
\begin{align}
	\sin (\alpha + \beta) & = \cos \alpha \sin \beta + \sin \alpha \cos \beta \\
	\sin (\alpha - \beta) & = \cos \alpha \sin \beta - \sin \alpha \cos \beta
\end{align}

\section{今週の作業内容}
今週行った作業内容を書く.
\subsection{作業1}
これも基礎理論と同様,わかりやすくsubsectionに分けてあげるべきである.

\subsection{参考文献の挿入}
参考文献は,\texttt{ref.bib}に書いておく.
参考文献を挿入する場合は,\texttt{cite}コマンドを使う.
例えば,

参考文献\cite{ohuchi2019}は光無線OFDMに関する解説論文である.

のように表示されるようになる.

\section{今後の課題}
今後の課題を書く.次の週にまず取り組むことを書くと良い.

% 参考文献
\bibliographystyle{ohuchi} % 大内研究室用のスタイルファイル
\bibliography{ref}        % 参考文献のファイル名 (デフォルトはref.bibに設定してあるけど,変えても良い)

\end{document}
