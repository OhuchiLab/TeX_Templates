% Version: 20241221

% プリアンブル (TeX ドキュメントの設定)=====

% ドキュメントクラスの設定: 左寄せ, A4用紙, 2段組
\documentclass[fleqn, a4paper, twocolumn]{jsarticle}

% パッケージの読み込み
\usepackage[dvipsnames]{xcolor}
\usepackage[dvipdfmx]{graphicx}
\usepackage{ascmac}
\usepackage{multicol}
\usepackage{physics, amssymb, amsmath}
\usepackage{enumitem}
\usepackage{titlesec}
\usepackage{titling}
\usepackage{here}
\usepackage{subcaption}
\usepackage{fancyhdr}

% TikZの設定 ---
\usepackage{tikz, braket}
\usetikzlibrary{patterns, intersections, calc, quotes, angles, arrows.meta, through, shapes.geometric}
\usetikzlibrary{shapes, positioning}

% 余白の設定 ---
\setlength{\mathindent}{3zw}
\setlength{\topmargin} {0cm}
\iftombow
    \addtolength{\topmargin}{0cm}
\else
    \addtolength{\topmargin}{-1truein}
\fi
\setlength{\textheight}   {45\baselineskip}
\addtolength{\textheight} {\topskip}
\setlength{\oddsidemargin}{-15pt}
\setlength{\textwidth}    {50zw}

% ページスタイルの設定 ---
\pagestyle{empty}
\lhead{}
\chead{}
\lfoot{}
\cfoot{--\;\thepage\;--}
\rfoot{}
\allowdisplaybreaks

% 図番号などの設定 ---
\numberwithin{figure}{section}
\numberwithin{table}{section}
\numberwithin{equation}{section}
\newcommand{\erfc}{\mathrm{erfc}}

% 資料のタイトルを入れる
\title{中間発表の題目}
% 資料の著者名を入れる
\date{夏季/冬季合同中間発表 2025年4月1日}
% ゼミの日を入れる
\author{名字 名前}

% =====

\begin{document}

% タイトル部分 (操作不要) =====
\twocolumn[
	\centering
	\noindent
	\textsf{\Large \thetitle\\}
	\thedate\\
	\theauthor (大内研究室)\\
]
% =====

% 本文 (ここから内容を記述)=====
% こういった長い文章は,別ファイルに記述して,\input{}で読み込むと良い.
% 読み込むときは,拡張子はいらない.ただ,章ごとに分けてもちゃんと
% ファイル名は.texと末尾につけてtexファイルであることを明示すること.
\section{はじめに}
\label{sec:introduction}   % このファイルのラベル こうしておくと別の章から参照しやすい.
ここには,研究の背景や目的,研究の意義などを書く.
大体A4で1ページ程度書けると良い.
\section{基礎理論}
\label{sec:fundamental_theory}
研究する上で必要な基礎理論を説明する.
例えば,伝送路推定方式の検討を行う場合,まずフェージングの種類の話をしたり,
想定する通信方式の話をしたりする.
\subsection{直交周波数分割多重}
直交周波数分割多重 (Orthogonal Frequency Division Multiplexing: OFDM) は,高速伝送を実現するための技術の一つである.OFDMには,以下の3つの要求条件が設定されている\cite{5gtextbook}. (ここからは省略)
\section{研究内容}
\label{sec:research}
ここでは,実際に研究内容を書いていく.
例えば,伝送路推定方式についての研究であれば,研究する伝送路推定方式について
説明する.
\section{まとめ}
\label{sec:summary}
これまで書いた内容をまとめる.
\section{今後の展望}
\label{sec:prospects}
今後やらなければならないこととかを書く.
% (ここまで)=====

% 参考文献
\bibliographystyle{ohuchi}  % 大内研究室用のスタイルファイル
\bibliography{ref}          % 参考文献のファイル名 (デフォルトはref.bibに設定してあるけど,変えても良い)

\end{document}